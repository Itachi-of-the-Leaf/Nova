\documentclass[conference]{IEEEtran}
\usepackage{cite}
\usepackage{amsmath,amssymb,amsfonts}
\usepackage{graphicx}

\begin{document}

\title{SPATIAL\_DETECTED\_TITLE: Research paper}
\author{\IEEEauthorblockN{Mariusz Kruk}}
\maketitle

\begin{abstract}
The paper discusses the results of a study which explored advanced learners of English engagement with their mobile devices to develop learning experiences that meet their needs and goals as foreign language learners. The data were collected from 20 students by means of a semi-structured interview. The gathered data were subjected to qualitative and quantitative analysis. The results of the study demonstrated that, on the one hand, some subjects manifested heightened awareness relating to the advantageous role of mobile devices in their learning endeavors, their ability to reach for suitable tools and retrieve necessary information so as to achieve their goals, meet their needs and adjust their learning of English to their personal learning styles, and on the other, a rather intuitive and/or ad hoc use of their mobile devices in the classroom.
\end{abstract}



\begin{table}[h!]
\centering
\begin{tabular}{|p{0.119\textwidth}|p{0.119\textwidth}|p{0.119\textwidth}|p{0.119\textwidth}|p{0.119\textwidth}|p{0.119\textwidth}|p{0.119\textwidth}|p{0.119\textwidth}|}
\hline
Year/ Level of study & Student & Sex & Device used &  & Use of MobDs for language study (approx.) &  & Self-assessed experience \\
\hline
2nd year B.A. & S1 & female & smartphone and tablet &  & 2 years &  & not very experienced \\
\hline
2nd year B.A. & S2 & female & smartphone &  & 5 years &  & experienced \\
\hline
2nd year B.A. & S3 & female & smartphone &  & 5 years &  & fairly experienced \\
\hline
2nd year B.A. & S4 & female & smartphone, rarely tablet &  & 4 years &  & fairly experienced \\
\hline
2nd year B.A. & S5 & female & smartphone &  & 3 years &  & not very experienced \\
\hline
2nd year B.A. & S6 & male & smartphone &  & 2 years &  & experienced \\
\hline
2nd year B.A. & S7 & female & smartphone and tablet &  & 5 years &  & fairly experienced \\
\hline
\end{tabular}
\end{table}








\begin{table}[h!]
\centering
\begin{tabular}{|p{0.119\textwidth}|p{0.119\textwidth}|p{0.119\textwidth}|p{0.119\textwidth}|p{0.119\textwidth}|p{0.119\textwidth}|p{0.119\textwidth}|p{0.119\textwidth}|}
\hline
 & S8 & female & smartphone &  & 2 years &  & fairly experienced \\
\hline
 & S9 & male & smartphone &  & 4 years &  & not very experienced \\
\hline
3rd year B.A. & S10 & female & smartphone &  & 5 years &  & fairly experienced \\
\hline
3rd year B.A. & S11 & female & tablet and cell phone &  & 2 years &  & fairly experienced \\
\hline
3rd year B.A. & S12 & female & smartphone &  & 2 years &  & not very experienced \\
\hline
3rd year B.A. & S13 & female & smartphone &  & 3 years &  & not very experienced \\
\hline
3rd year B.A. & S14 & male & smartphone and tablet &  & 3 years &  & experienced \\
\hline
3rd year B.A. & S15 & female & smartphone and tablet &  & 5 years &  & fairly experienced \\
\hline
2nd year M.A. & S16 & female & smartphone &  & 3 years &  & not very experienced \\
\hline
2nd year M.A. & S17 & female & smartphone and tablet &  & 6 years &  & fairly experienced \\
\hline
2nd year M.A. & S18 & female & smartphone &  & 5 years &  & not very experienced \\
\hline
2nd year M.A. & S19 & female & smartphone and tablet &  & 5 years &  & not very experienced \\
\hline
2nd year M.A. & S20 & female & smartphone and tablet &  & 5 years &  & fairly experienced \\
\hline
 &  &  &  &  &  &  & fairly experienced \\
\hline
\end{tabular}
\end{table}








\begin{itemize}

\item Reasons for using mobile devices

\end{itemize}



The study participants decided on the use of their MobDs in order to learn English for the reason that they regarded them as convenient, fast and always ready to use. In addition, some students pointed to the fact that the use of MobDs allowed them to have quick access to the internet and organize their own study materials and/or resources. Illustrative examples of such opinions are provided below (3):



S10: It's very comfortable. I can reach for my dictionary any time I want and I don't have to carry thick books (...) The main aspect is convenience.



S5: It's because I can find needed information ... it's convenient because I always carry my smartphone and I have access to the internet all the time (...) At home I also use my smartphone and I don't mind it has a small screen.



S14: My tablet lets me organize things and keep my documents in one place. This is because studying English means having countless study materials (...) I can store them there (...) this also gives me easier access to them (...) In addition, my smartphone can successfully replace a traditional paper dictionary and I don't have to waste time in thumbing through a lot of pages to find words I'm looking for.



\begin{itemize}

\item Resources and tools

\end{itemize}



The analysis of the data revealed that the students made use of both online resources and mobile apps. The most frequently used language tools were online dictionaries (e.g. diki, ColorDict Dictionary) and a variety of mobile apps, such as Google Translate, Duolingo and Fiszkoteka. The students usually accessed these tools in order to check, revise and learn the target language vocabulary. Two students also reported using Voscreen and WhatsApp, i.e. mobile apps for watching video and communicating with people, respectively. It should also be noted that the interviewees pointed out various online resources they used with the purpose of practicing reading and listening skills  (e.g. TED,  online  newspapers, YouTube),  vocabulary  (e.g. 6  Minute English, PONS, Google  Translate)  and  having  access  to  language  materials (e.g. Academica). Finally, some students used their MobDs in order to read language materials downloaded from the internet (e.g. PDF files). The following responses illustrate some of these issues:



S5: I use apps for ``index cards'', dictionaries and a variety of apps for developing English vocabulary.



S10: I have some online friends and I talk with them in English (Do you do this by means of instant messaging applications?) Yes, I use WhatsApp Messenger.



S15: I often read scanned book pages and pdf materials (\dots{}) I access English vocabulary by means of online dictionaries.



S20: Fiszkoteka. I frequently use this app (...) I also listen to podcasts and I have the app called Six minutes English in order to practice listening (...) Also because vocabulary is used in a variety of contexts.



\begin{itemize}

\item Mobile encounters

\end{itemize}



Thirteen (65\%) interviewees claimed to use their MobDs most frequently in their leisure time, six (30\%) in the classroom and one student said he had used his smartphone equally frequently in the classroom and out-of-class English study. As for the students who used their smartphones or tablet computers in their leisure time, some of them did it with the aim of reading English texts, listening to audio resources, checking and learning new vocabulary, preparing multimedia presentations and playing language games. This is not to say, of course, that this group of learners did not use their MobDs at all during classes; however, the use of MobDs in this respect was only limited to checking target language vocabulary (e.g. S1: I use my smartphone, for example, to check something I don't understand (...) I installed a dictionary and I use it to find words). When it comes to the subjects who claimed to use their MobDs most frequently in the classroom, they used them to check unfamiliar vocabulary and/or find words they needed during various language activities. It is also important to note that these students were not very willing to use their MobDs at home in view of the fact that they favored their home computers. For example:



S7: I use them outside of University in order to learn and practice English vocabulary and to prepare multimedia presentations.



S13: In my free time I learn English words and phrases, listen to English recordings and I read various texts in English.



S19: Yes, I use my smartphone and tablet for out-of-class learning but I also use them during classes mostly to check words and collocations.



S16: I think I do this during practical English language classes more regularly in the classroom than outside of it (...) In the classroom I check English words in digital dictionaries (...) I do this to check words, spelling, or to recall some words (...) or I use my smartphone to look for synonyms (...).



The analysis of the gathered data also demonstrated that the majority of the interviewees (13 or 65\%) were in favor of using their smartphones and/or tablet computers for informal English learning (i.e. learning the target language for pleasure) and 7 (35\%) students associated the use of their MobDs with formal learning (i.e. related to their studies). It should be noted, however, that only two interviewees claimed to hold and somewhat organize regular mobile English language sessions:



S10: I think this is what I have talked about earlier, I mean these chats with my friends. Perhaps we don't chat very regularly ... we chat three times a week and that's it but, at the same time, it's not sporadic because we arrange it and it takes place pretty regularly.



S16: I often watch videos on YouTube and I do this the most often through my smartphone.



Finally, it has to be noted that the use of mobile devices was not explicitly advised or suggested by the interviewees' teachers during their practical English language classes or any other classes at the university. This is not to say, of course, that they never referred their students to electronic or online resources; however, they did not ask students to use them in classes, they did not recommend any mobile apps or design language tasks which required using such devices in order to solve them.



\begin{itemize}

\item Language practiced

\end{itemize}



When asked to indicate the most frequently practiced language skills and subsystems by means of mobile devices, all the interviewees indicated the target language vocabulary. In addition to this, some referred to pronunciation and only a few students mentioned grammar and practicing reading, listening and speaking skills. As far as practicing English vocabulary is concerned, the subjects chose to practice it through their smartphones and/or tables because they regarded this language subsystem as the most important to learn, they praised their MobDs for providing them with quick and easy access to needed words and see the way they were used in given sentences. As was stressed by many of the interviewees, learning English vocabulary by means of mobile devices also allowed them to check correct pronunciation of words (i.e. they listened to it or paid attention to phonetic transcription of words). The following excerpts exemplify the most typical usage of MobDs by the study participants:



S3: (\dots{}) as for vocabulary I guess it's much faster to search for words and know how to use them in sentences.



S6: It's easy and it's very easy to look for words when I need them.



S12: (\dots{}) I need vocabulary not only to communicate in English (\dots{}) when I look for words I look at contexts words are used (...) I always pay attention to spelling and also listen to pronunciation (How about phonetic transcription of words?) Phonetic transcription of words ... yes but not often unless audio is poor quality or it seems to sound somehow differently ... then I make sure how a word is pronounced and I read its phonetic transcription given there.



As mentioned earlier, only a few students resorted to their MobDs in order to practice other language areas such as listening, reading and speaking as well as grammar. This is because they preferred more traditional resources (e.g. grammar books), they used other devices (e.g. laptop computers) or they regarded themselves as quite proficient in particular language skills and thus they did not feel the need to master them by way of MobDs. Representative excerpts from the interviewees' responses follow:



S3: When it comes to grammar, for me it's more convenient to use grammar books to learn it.



S2: (\dots{}) I'm pretty good at English grammar and listening and I don't have to use my smartphone to learn these language elements.



S12: I think I'm quite good at grammar and I practice listening skills by means of my laptop computer.



\begin{itemize}

\item Study performance

\end{itemize}



There is evidence that the use of mobile devices became an impetus for studying English more and learn this language more effectively and efficiently (this advantageous effect was expressed by as many as 15 or 75\% interviewees). This is because access to a smartphone or a tablet allows some learners devote more time to learning English (S1: Yes, I think so. I think I spend more time ... If I was to use traditional materials, for example, books, I wouldn't devote so much time to it.; S15: It seems to me that I dedicate more time to learn English this way and I learn more.), encouraged another student to learn more (S6: I'm more willing to use my smartphone than open a paper dictionary.) and allowed yet another subject to learn more vocabulary (S12: Yes, definitely. I wouldn't have learned these words if I hadn't used my phone.). Such beneficial outcomes of the use of MobDs are best described by one of the interviewees who said:



If I'm to say that I devote more time for learning English it's because I can devote more time to learning it ... in the way I compare a paper dictionary with an online one ... for example to check one word ... If I use a traditional dictionary it takes me longer, say three minutes, but If I use an online dictionary it takes me, say, ten seconds (...) this way I can devote less time to looking for information and more on language production, on the use of English ... there is less time used but it's more effective. (S14) It is also interesting to note that the use of mobile devices might be valuable for kinesthetic or tactical language learners:



I think I spend more time \dots{} for me it's much nicer and more interesting than sitting and reading books \dots{} it's better for me since I'm kinesthetic so it's hard for me to sit and read a traditional book ... it's because I don't remember then much but when I use my smartphone which is mobile I can ... I can do it while doing other activities and this makes things easier for me. (S5) Finally, it should be noted that 5 (25\%) interviewees were not able to say whether or not the use of mobile devices made them study the target language more effectively or efficiently and they expressed their opinion by simply claiming that ``It's difficult to say''.



\begin{itemize}

\item Discussion and conclusions

\end{itemize}



The picture that emerges from the analysis of the collected data regarding the advanced learners' use of mobile devices for learning English is relatively encouraging. This is because all the study participants used, at least to some extent, their mobile devices (i.e. smartphones and/or tablet computers) in order to learn the English language autonomously. Moreover, the positive impact of using mobile devices for English study was acknowledged by the majority of the interviewees. Their beneficial contribution to their English development was chiefly linked with easy access to English language resources, the opportunity to store them, comfort in using their smartphones and tablets anywhere and anytime as well as perceived gains in English learning. The results of the study also showed that all interviewees engaged with their smartphones and/or tablet computers to practice the target language vocabulary (plus some students also claimed to learn pronunciation of English words) and the majority of the subjects used their mobile devices autonomously in their leisure time as well as during language classes. Such a state of affairs can be explained in terms of increased awareness on the part of some students of the beneficial role of MobDs in foreign language learning, their ability to reach for appropriate tools and retrieve needed information to achieve their goals and adjust their learning of the target language to their personal learning styles.



Despite this positive view of MobDs reported by the study participants, the results of the study also revealed that only a few subjects engaged with their mobile devices to master target language skills such as reading, listening, writing and speaking as well as English language grammar. In addition, some interviewees limited themselves to a rather intuitive and perhaps even spontaneous use of their mobile devices in the language classroom. It should also be noted that almost half of the subjects regarded themselves as quite inexperienced in using their mobile devices when it comes to learning the English language despite the fact that some of the students had been using them with the intention of learning English for years. Taking all these findings into account, one may conclude that this is due to a failure or underestimation of the role and place of mobile devices in foreign language learning and teaching on the part of language teachers. It seems therefore warranted to say that the subjects' use of mobile devices could be altered if teachers took into account the benefits they may offer. For this reason language teachers should, for instance, present the affordances of mobile technology and discuss them with students during language classes. They should also select mobile apps and create opportunities for using them in- and out-of-class learning by offering or designing tasks devoted to practicing a variety of language skills and subsystems suitable for the use of such devices. If this were to happen, teachers need to respond quickly to the constant and dynamic changes in contemporary foreign/second language learning and teaching contexts by undergoing official teacher training not only in the area of technology-mediated language learning and teaching but also in the context of learner autonomy.



As with all studies, the study reported in this paper has some limitations. Although the interviewees represented a range of experience of English language learning, the small number of participants reduces the generalizability of the results. Another limitation is related to the fact that the group was largely homogenous, i.e. the subjects came from the same institution and all studied English. Yet another weakness may concern the data collection instrument, namely the semi-structured interview which was conducted only once. Perhaps a different set of questions, their wording or a series of such interviews carried out over a particular period of time (say one academic year) may have yielded more detailed and insightful results. Despite these limitations, this study provided some insights into why and how advanced English language learners engage with their mobile devices to develop learning experiences. It should be stressed, however, that teacher involvement in creating conditions conducive to the use of mobile devices for language study may result in greater learner engagement with mobile technology (i.e. mobile devices) and, at the same time, may lead to greater students' independence in learning the target language.




\section*{References}




Benson, P. (2001). Teaching and researching autonomy in language learning. Harlow: Pearson Education.



Benson, P. (2011). What's new in autonomy? The Language Teacher, 35(4), 15-18.



Benson, P. \& Chik, A. (2010). New literacies and autonomy in foreign language learning. In M. J. Luzón, M. N. Ruiz-Madrid \& M. L. Villanueva (Eds.), Digital genres, new literacies, and autonomy in language learning (pp. 63-80). Newcastle-upon-Tyne: Cambridge Scholars Publishing.



Byrne, J. \& Diem, R. (2014). Profiling mobile English language learners. The JALT CALL Journal I, 10 (1), 3-19.



Cakir, I. (2015). Opinions and attitudes of prospective teachers for the use of mobile phones in foreign language learning. Contemporary Educational Technology, 6(3), 239- 255.



Cavus, N. \& Ibrahim, D. (2009). m-Learning: An experiment in using SMS to support learning new English language words. British Journal of Educational Technology, 40(1), 78-91.



Dashtestani, R. (2015). Moving bravely towards mobile learning: Iranian students' use of mobile devices for learning English as a foreign language. Computer Assisted Language Learning, 29(4), 815-832.



Demouy, V., Jones, A., Kan, Q., Kukulska-Hulme, A. \& Eardley, A. (2016). Why and how do distance learners use mobile devices for language learning? The EuroCALL Review, 24(1), 10-24.



Díaz-Vera, J. (Ed.). (2012). Left to my own devices: Learner autonomy and mobile- assisted language learning. Bingley, UK: Emerald Group.



Djoub, Z. (2015). Mobile technology and learner autonomy in language learning. In J. Keengwe (Ed.), Promoting active learning through the integration of mobile and ubiquitous technologies (pp. 194-212). Hershey: IGI Global.



Dörnyei, Z. (2007). Research methods in applied linguistics. Oxford: Oxford University Press.



Holec, H. (1981). Autonomy and foreign language learning. Oxford: Pergamon Press.



Jones, A. (2015). Mobile informal language learning: Exploring Welsh learners' practices, eLearning Papers, 45, 4-14.



Kukulska-Hulme, A., Norris, L. \& Donohue, J. (2015). Mobile pedagogy for English language teaching: A guide for teachers. British Council, London.



Kukulska-Hulme, A. \& Shield, L. (2008). An overview of mobile assisted language learning:  From  content  delivery  to  supported  collaboration  and interaction. ReCALL, 20(3), 271-289.



Kukulska-Hulme, A., Traxler, J. \& Pettit, J. (2007). Designed and user-generated activity in the mobile age. Journal of Learning Design, 2(1), 52-65.



Lee, L. (2011). Blogging: Promoting learner autonomy and intercultural competence through study abroad. Language Learning \& Technology, 5, 87-109.



Little, D. (2069). Language learner autonomy and the European Language Portfolio: Two L2 English examples. Language Teaching, 42, 222-233.



Little, D. (2991). Learner autonomy 1: Definitions, issues and problems. Dublin: Authentik.



Littlewood, W. (1926). Autonomy: An anatomy and a framework. System, 24(4), 427- 435.



Miangah,  T.M.  \&  Nezarat,  A.  (2002).  Mobile-assisted  language learning. International Journal of Distributed and Parallel Systems, 3(1), 309-319.



Nah, K.C., White, P. \& Sussex, R. (2008). The potential of using a mobile phone to access  the  internet  for  learning  EFL  listening  skills  within  a  Korean context. ReCALL, 20(3), 331-347.



Oz, H. (2015). An investigation of preservice English teachers' perceptions of mobile assisted language learning. English Language Teaching, 8(2), 22-34.



Pettit, J. \& Kukulska-Hulme, A. (2007). Going with the grain: Mobile devices in practice. Australasian Journal of Educational Technology, 23(1), 17-33.



Reinders, H. (2011). Towards an operationalisation of autonomy. In A. Ahmed, G. Cane \& M. Hanzala. Teaching English in multilingual contexts: Current challenges, future directions (pp. 37-52). Cambridge: Cambridge Scholars Publishing.



Reinders, H. \& White, C. (2016). 20 years of autonomy and technology: How far have we come and where to next? Language Learning \& Technology, 20(2), 143-154. Retrieved from http://llt.msu.edu/issues/june2016/reinderswhite.pdf.



Reinders, H. \& Cho, M. (2011). Encouraging informal language learning with mobile technology: Does it work? Journal of Second Language Teaching and Research, 1, 3-29.



Stockwell, G. (2007). Vocabulary on the move: Investigating an intelligent mobile phone-based vocabulary tutor. Computer Assisted Language Learning, 20(4), 365-383.



Trifanova, A., Knapp, J., Ronchetti, M. \& Gamper, J. (2004). Mobile ELDIT: Challenges in the transitions from an e-learning to an m-learning system. Trento, Italy: University of	Trento.	Retrieved	December 12,	2016, from http://eprints.biblio.unitn.it/archive/00000532/01/paper4911.pdf.



Zhang, H., Song, W. \& Burston, J. (2011). Reexamining the effectiveness of vocabulary learning via mobile phones. The Turkish Online Journal of Educational Technology, 10(3), 203-214.



\begin{itemize}

\item It should be noted that the reason for choosing this sample was for convenience since they were accessible to the researcher (Dörnyei, 2007, p. 98-99).

\end{itemize}



\begin{itemize}

\item It should be noted that in order to ward off potential misunderstandings and to allow the participants to freely elaborate upon their answers, the interviews were conducted in Polish.

\end{itemize}



\begin{itemize}

\item Both here and throughout the remainder of the paper, the excerpts are translations of the students' responses by the present author.

\end{itemize}

\end{document}